% \documentclass[letterpaper]{article}
\documentclass[14pt]{extarticle}
\usepackage{polski}
% \usepackage[hebrew]{babel}
\usepackage[utf8x]{inputenc}
\usepackage[letterpaper, margin=1in]{geometry}

\title{Tashlikh}
\author{Franciszek Malinka}
\date{Styczeń 2022}
\begin{document}
\maketitle
Holokaust, rzeź Ormian, ludobójstwo Serbów, czystki w~kolonialnych państwach Afryki -- Ya'el Bartana w~swoim krótkometrażowym filmie pt. \textit{Tashlikh} konfrontuje traumatyczne wspomnienia ofiar i~oprawców największych masakr XX wieku. Osobiste przedmioty, narzędzia zbrodni i~symbole religijne, rzucone w~bezgraniczną przestrzeń rozszerzają tradycyjny rytuał \textit{taszlich}, w~którym drobiny chleba wrzucane są do morza w~celu oczyszczenia z~grzechów.  

Dzieło jest filmem trwającym 11 minut 14 sekund, przedstawiającym zbiór różnorakich przedmiotów, opadających w~zwolnionym tempie na~czarnym tle. Przez większość trwania filmu słychać niski, rytmiczny dźwięk, który przepleciony jest przenikliwym szumem, odliczaniem w~języku niemieckim czy rykiem silników samolotów.

Początkowe przedmioty, które widzimy na~ekranie to~kamizelki bezpieczeństwa, plecak, duża torba, klapki, dżinsy, żółta sportowa koszulka, zawinięte koce. W tym momencie pojawia się wolno opadająca chusta w biało-czerwony wzór, a za nią szybko opadające karabiny AK. Następnie widzimy biały kanister, materiałową torbę z warzywami, za którą widać spadające dynie i pęk marchewek, dużo drobin, najprawdopodobniej kawałki chleba, między którymi opadają niemieckie paszporty i małe klucze. Zaraz za nimi opadają kolejne paszporty i klucze, w tym jeden duży, złoty klucz. Następnie po prawej stronie ekranu widoczna jest ręcznie robiona lalka o czarnym namaszczeniu, po lewej opadają ziemniaki, które wypadają z worka. Powoli nasila się niski dźwięk. Pojawia się stara proteza oraz kula. Wraz z nasilaniem się dźwięku zaczynamy słyszeć wyższy, niepokojący, przeszywający dźwięk, a w tym czasie widać jedynie parę spodni ozdobionych ludowymi wzorami. Zaraz za nimi opadają obok siebie dwie figurki, jedna przedstawiająca Statuę Wolności, druga posąg Dawida. Za nimi widzimy pusty futerał do skrzypiec oraz część munduru i wolno opadającą oficerską czapkę. Przez kilka sekund nie widać żadnego przedmiotu, a szum osiąga kulminacyjny moment. Pojawiają się skrzypce, szybko opadający pistolet, flaga Jugosławii, dwie róże oraz żydowski talit. Wraca rytmiczny, niski dźwięk i przez kilka sekund nie widać żadnego przedmiotu.

W centralnym punkcie ekranu widać sportową kurtkę Adidas, po czym ponownie bardzo powoli opada talit i słychać dźwięk tłuczonego szkła. Po chwili widać menorę i ramkę z potłuczoną szybką ze zdjęciem żołnierza. Przez czas opadania talitu słychać dziecko mówiące niemieckie liczby. Ma się wrażenie, jakby głos dziecka dobiegał ze słuchawki telefonu. Oddzielone krótką przerwą pojawiają się damskie ubrania ludowe, a w tle słychać szum przypominający pociąg jadący po torach. Po lewej stronie szybko spadają monety, a później banknoty. Widać dżinsy i kolejne ludowe ubrania, a przez środek szybko spada ludzka czaska, której towarzyszy głośny szum. Znów słychać dziecko liczące po niemiecku, spada kolejna czaszka, menora. Po lewej stronie spada pasiasta koszula i spodnie, a po prawej otwarta walizka, z której wypadają maskotki, czarno-białe zdjęcia i lalki. W górnej części ekranu, na moment, można było dostrzec ludzką dłoń, która jednak nie zacząła opadać tak jak reszta przedmiotów. W tym czasie pojawiają się dźwięki skrzypiących drzwi, piski oraz nasilający się, pulsujący alarm. Następnie widzimy dużą lalkę, kolejne zdjęcia, zaraz za którymi, po lewej stronie, dostrzegamy pas dużej amunicji, obok której spada album, w którym widać zdjęcie czarnoskórego mężczyzny oraz plemienna maska. Oddzielone krótką przerwą opadają czaszka, smyczek, żołnierski hełm, książka, ramka z obrazem kobiety, trzy pary okularów, dwa pistolety, jeden stary i jeden nowoczesny. Słychać dźwięk tłuczenia, po którym następuje opadający, roztrzaskany, gliniany wazon, okulary i złoty wisior. 

W całkowitej ciszy, bardzo powoli, z góry ekranu wyłania się koc termiczny, za którym szybko spada maskotka przedstawiająca psa bez jednego ucha. Koc jest centralnym elementem i zajmuje większość ekranu. W tle słychać bardzo delikatne skrzypienia, a następnie spada bardzo duża ilość okruchów oraz czerwona spódnica. Skrzypienia, w porównaniu do poprzednich dźwięków, są delikatniejsze, a nawet kojące. Widać opadające kawałki gazet z niemieckimi napisami, wąskie książki. Pojawia się szybko spadająca figurka statku, obraz archanioła Michała, dwa świeczniki, guzik z symbolem \textit{SS}, dwa ordery i złoty zegarek. Przez ułamek sekundy ekran jest pusty.

W tle słychać bardzo powoli narastający szum. W środku ekranu pojawia się żółta, sportowa koszulka z logiem \textit{Orange} oraz flagą Kamerunu. W tym czasie po lewej stronie opadają figurki z szopki, tj. Maryi, Józefa, Jezusa w żłobie i trzech króli. Oddzielone przerwą pojawiają się subha (muzułmańskie sznury modlitewne) oraz książka, prawdopodobnie Koran. Za nimi spadają czerwona płachta, czapka oficerska, ramka ze zdjęciem mężczyzny na tle flagi Tureckiej. Szum jest coraz głośniejszy. Po środku pojawia się pognieciona, metalowa butelka, z której wylatują krople wody. Następnie widać okładkę płyty zatytuowanej \textit{Sing mit Pionier}, która otoczona jest mnóstwem małych, złotych gwiazd Dawida. Spada coraz więcej gwiazd, a szum staje się bardzi głośny i brzmi jak ryk silników odrzutowych. W centralnej części opadają same dziecięce buciki i niebieski płaszczyk. Kiedy te znikają, pojawia się bardzo dużo czarno-białych zdjęć, książka, biały wazon w niebieskie wzory, ramka ze zdjęciem. Wraca niskie dudnienie oraz pulsujący dźwięk. Pojawia się puszka z napisem (w tłumaczeniu) \textit{Żydowski Fundusz Narodowy}, książka, banknoty, białe talerze z niebieskimi zdobieniami. Następnia opada czapka oficera Wermachtu, małe figurki samolotów bojowych, czarne oficerskie buty, stary, pożółkły zeszyt, okulary. 

W centralnej części pojawia się biało-czarna chusta, w ten sam wzór, co biało-czerwona chusta na początku filmu. Zajmuje centralną część ekranu. Ostatnimi przedmiotami, które widzimi są białe pantofelki, niebieski damski kapelusz oraz bardzo duży, złoty klucz, ten sam co przy początku filmu. Klucz spada bardzo powoli i~jako ostatni znika z~ekranu. Szum silnika powoli ustaje, pozostaje niskie dudnienie. Przez ostatnie sekundy filmu słychać jedynie dźwięk, nie pojawia się żaden przedmiot.

Czarne tło i~spowolnione tempo opadania przedmiotów sprawia wrażenie, że akcja filmu dzieje się w~przestrzeni kosmicznej albo w bezkresnej podwodnej odchłani. Jednakże kamizelka ratunkowa, koc termiczny, walizki czy odgłosy silników mogą również wskazywać, że doszło do katastrofy lotniczej, a~przedmioty spadają z~wraków samolotu. Stawia to przed nami pytanie -- kim byli pasażerowie? 

Artystka tworzy iluzję oderwanej od naszego świata przestrzeni, w~której opadające przedmioty, mimo swojego osobistego przeznaczenia, nie mają bezpośredniego połączenia z~konkretnymi wydarzeniami czy osobami. Każda z występujących rzeczy staje się symbolem traumy, ale też piętna. Pojawiające się rekwizity religijne, ludowe ubrania, zdjęcia, paszporty, symbole narodowe; wszystkie te przedmioty są kluczem do rozszyfrowania tożsamości pasażerów tego zniszczonego samolotu.

Przez całą długość filmu przeplatane są atrybuty oprawców razem z symbolami związanymi z konkretnymi narodowościami czy grupami etnicznymi. Przedstawiona jest odzież z ludowymi symbolami Palestyńczyków (\textit{Nakba}), flaga Jugosławii (ludobójstwo Serbów), judaistyczne symbole religijne (Holokaust), plemienna maska i czarna lalka (ludobójstwo Herero, Tutsi) -- wszystkie te elementy nawiązują do tragicznej historii narodów, które łączy uciemiężenie na tle dyskryminacji etnicznej. Dostrzegamy również osobiste przedmioty oprawców -- zdjęcia, ubrania, gazety, paszporty. Wskazuje to na to, że na pokładzie tego samolotu kaci siedzieli obok swoich ofiar.

Maska, dziecięca laska i pas amunicji; okulary, zeszyty, zdjęcia, mundury żołnierzy Wermachtu; symbole kultu i karabiny maszynowe: Bartana grą dźwięku, spowolnionego obrazu i kontrastową kompozycją przedmiotów tworzy dysonans, który potężnie rezonuje w naszej wyobraźni. Przez całą długość filmu czujemy niepokój, głośny i pulsujący dźwięk świdruje nasz umysł. Trudno jest objąć słowami ogrom bólu, okrucieństwo i skalę tragedii, jaka spotkała eksterminowane ludy w ubiegłym stuleciu. Artystka, bez użycia mowy, przekazuje te emocje przy pomocy efektów audiowizualnych.

Tytuł filmu, \textit{Tashlikh}, jest nazwą judaistycznego obrzędu oczyszczenia z~grzechów, podczas którego czytany jest tekst księgi Micheasza, z~której wywodzi się ten rytuał. Stanowi to główną wskazówkę w oczytaniu treści filmu. Przedmioty spadające z samolotu, mieszające się relikty przeszłości, narzędzia krwawych zbrodni, zdjęcia małżonków, ubrania dzieci, wszystkie wrzucone w jedną przestrzeń, rozszerzają tradycyjny żydowski rytuał. Akt wyrzucenia wspomnień staje się \textit{katharsis} dla zarówno ofiar, jak i oprawców. 

\begin{center}
\textit{\textbf{18} Któryż Bóg podobny Tobie,\\
co oddalasz nieprawość,\\
odpuszczasz występek\\
Reszcie dziedzictwa Twego?\\
Nie żywi On gniewu na zawsze,\\
bo upodobał sobie miłosierdzie.\\
\textbf{19} Ulituje się znowu nad nami,\\
zetrze nasze nieprawości\\
i wrzuci w głębokości morskie\\
wszystkie nasze grzechy.\\
(Księga Micheasza 7, 18-19)}
\end{center}

Reżyserka filmu nie namawia do zapomnienia o katastrofach, które miały miejsce w ubiegłym wieku, wręcz przeciwnie -- tworząc taki film manifestuje pamięć o ofiarach i zbrodniach dokonanych przez ludzi zgubionych przez ideologię, która popycha ich do mordowania milionów ludzi. Mimo tego prosi, żeby wspólnie ze swoimi prześladowcami wspólnie odbyć rytuał oczyszczenia.

Film, można oglądać wielokrotnie, za każdym razem odkrywając nowe znaczenia i symbole. Niesposób dostrzec wszystkie znaczenia i treści, które autorka poukrywała w przedstawionych przedmiotach. Mimo niepokoju, a może i strachu, który czuje się oglądając tę produkcję, chce się obejrzeć ją do końca i próbować to robić jeszcze wiele razy, tylko po to, żeby lepiej zrozumieć historię stojącą za kompozycją Bartany. 


% Ktoś mógłby powiedzieć, że nie da wybaczyć popełnienia zbrodni takiego formatu, że przeprowadzenie takiego obrzędu nie jest realne. Jednakże, czy w jednoznaczny sposób możemy podzielić narody na te atakowane i atakujące? Czy spadające w bezkres przedmioty są podzielone między te należące do ofiar i te należące do oprawców? Artystka odpowiada na te pytania. Aby to dostrzec należy przytoczyć pobudki popchnęły do działania morderców. Nietrudno zauważyć, że w każdej rzezi, której historia przekazana jest w filmie, jedną z najważniejszych cech wspólnych jest dyskryminacja na tle religijnym. Jednakże ta dyskryminacja tworzy koło -- Żydzi wypędzają muzułmańskich Palestyńczyków; muzułmańskie Imperium Osmańskie dokonuje rzezi na chrześcijańskich Ormianach; katoliccy Chorwaci mordują prawosławnych Serbów; Serbowie zabijają bośniackich muzułmanów.

% \begin{thebibliography}{9}
% \bibitem{taszlich}
% https://www.myjewishlearning.com/article/tashlikh/
% \bibitem{biblia}
% Biblia Tysiąclecia https://biblia.deon.pl/
% \end{thebibliography}

\end{document}